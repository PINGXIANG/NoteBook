%%
%% All packages and macros needed for the problemsets
%%
\usepackage{amsmath}

\usepackage{lipsum}
%\usepackage{showframe}
%\usepackage{layout}

% Chinese 
\usepackage{xeCJK}

% Mathbf 
\usepackage{bm}

\usepackage[charter,cal=cmcal]{mathdesign} %different font
%\usepackage{avant}

% for emphasize
\usepackage{ulem}
\makeatletter
\newcommand\colorwave[1][red]{\bgroup \markoverwith{\lower3.5\p@\hbox{\sixly \textcolor{#1}{\char58}}}\ULon}
\font\sixly=lasy6 % does not re-load if already loaded, so no memory problem.
\makeatother

% for dash box
\usepackage{adjustbox}
\usepackage{dashbox}

% for cancel line style in STA
\usepackage[makeroom]{cancel}

% make hyperlink a different color
\usepackage{hyperref}
% def the refer color
\hypersetup{
    colorlinks=true,
    linkcolor=red,
    filecolor=magenta,      
    urlcolor=blue,
}

% footnote number by page
\usepackage[perpage]{footmisc}

% \floor and \ceil function
\usepackage{mathtools}
\DeclarePairedDelimiter\ceil{\lceil}{\rceil}
\DeclarePairedDelimiter\floor{\lfloor}{\rfloor}

% not package
\usepackage{centernot}


\usepackage{microtype}

\usepackage{etoolbox}
%\usepackage{amsfonts}
%\usepackage{amssymb}
\usepackage{graphicx}
\usepackage[inline]{enumitem}
\usepackage{xparse}
\usepackage{ifthen}
\usepackage{graphicx}
\usepackage{caption}
\usepackage{subcaption}
\usepackage{color}
\usepackage{tikz}
	\usetikzlibrary{fit}
	\usetikzlibrary{fadings}
	\usetikzlibrary{calc}
	\tikzset{>=latex}
	\usetikzlibrary{cd}
	\usetikzlibrary{spy}
\usepackage{fancyhdr}
\usepackage{calc}
\usepackage{wrapfig}
\usepackage{marginnote}
\usepackage{mparhack}
\usepackage{marginfix}
%%% Jason's hyperref
% \usepackage[hidelinks]{hyperref}
% \usepackage{fnpct} % fancy footnote spacing
% \usepackage{bm}
% \usepackage{systeme}
% \usepackage{datatool}% http://ctan.org/pkg/datatool for sorted lists
\usepackage{xspace}



\usepackage{pgfplots}
\pgfplotsset{compat=newest}
	\usepgfplotslibrary{fillbetween}
%%%
% Useful Linear Algebra macros
%%%
\newcommand{\declarecommand}[1]{\providecommand{#1}{}\renewcommand{#1}}
\declarecommand{\R}{\mathbb{R}}  % we don't care if it's already defined.  We really want *this* command!
\declarecommand{\Z}{\mathbb{Z}}
\declarecommand{\Q}{\mathbb{Q}}
\declarecommand{\N}{\mathbb{N}}
\declarecommand{\C}{\mathbb{C}}
\declarecommand{\d}{\mathrm{d}}
\declarecommand{\dd}{\mathbbm{d}} % exterior derivative
\DeclareMathOperator{\Span}{span}
\DeclareMathOperator{\Img}{img}
\DeclareMathOperator{\Id}{id}
\DeclareMathOperator{\Ident}{\Id}
\DeclareMathOperator{\Vol}{Vol}
\DeclareMathOperator{\VolChange}{Vol\hspace{1.5pt}Change}
\DeclareMathOperator{\Range}{range}
\DeclareMathOperator{\Rref}{rref}
\DeclareMathOperator{\Rank}{rank}
\DeclareMathOperator{\Comp}{\Vcomp}
\DeclareMathOperator{\Vcomp}{v\hspace{1pt}comp}
\DeclareMathOperator{\Null}{null}
\DeclareMathOperator{\Nullity}{nullity}
\DeclareMathOperator{\Char}{char}
\DeclareMathOperator{\Proj}{proj}
\DeclareMathOperator{\Flux}{Flux}
\DeclareMathOperator{\Circ}{Circ}
\DeclareMathOperator{\chr}{char}
\DeclareMathOperator{\Dim}{dim}
\DeclareMathOperator{\Perp}{perp}
\DeclareMathOperator{\Ker}{kernel}
\DeclareMathOperator{\Row}{row}
\DeclareMathOperator{\Col}{col}
\DeclareMathOperator{\Rep}{Rep}
\newcommand{\BasisChange}[2]{[#2\!\leftarrow\!#1]}
\newcommand{\proj}{\Proj}
\newcommand{\rref}{\Rref}
\newcommand{\xhat}{{\vec e_1}}
\newcommand{\yhat}{{\vec e_2}}
\newcommand{\zhat}{{\vec e_3}}
\newcommand{\sbasis}[1]{\vec { e}_{#1}}


\newcommand{\formarg}[2]{\big(#1;\, #2\big)}
\DeclarePairedDelimiter\abs{\lvert}{\rvert}
\DeclarePairedDelimiter\Abs{\lvert}{\rvert}
\DeclarePairedDelimiter\norm{\lVert}{\rVert}
\newcommand{\Norm}[1]{\norm{#1}}
% just to make sure it exists
\providecommand\given{}
% can be useful to refer to this outside \Set
\newcommand\SetSymbol[1][]{%
	\nonscript\::%
	\allowbreak
	\nonscript\:
	\mathopen{}}
\DeclarePairedDelimiterX\Set[1]\{\}{%
	\renewcommand\given{\SetSymbol[\delimsize]}
	#1
}



% labels for source attributions
\NewDocumentCommand{\beezer}{o}{%
	\IfNoValueTF{#1}{%
		{\color{blue}\sffamily{B}}%
	}{%
		{\color{blue}\sffamily{B}}%  XXX Todo, make this href to the appropriate problem number
	}\xspace%
}
\NewDocumentCommand{\hefferon}{o}{%
	\IfNoValueTF{#1}{%
		{\color{blue}\sffamily{H}}%
	}{%
		{\color{blue}\sffamily{H}}%  XXX Todo, make this href to the appropriate problem number
	}\xspace%
}

%% Package for notation
\usepackage{xstring}
