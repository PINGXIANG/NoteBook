\section{The axiom of completeness}
\begin{notation}
    \item $a_0 \in \mathcal{A} \in \R$ is \uline{maximum of A} \iff $a_0 > a, \any a \in \mathcal{A}$.
    \footnote{The biggest difference between $sup \mathcal{A}$: $a_0 \in \mathcal{A}$.}
    \item $c + A := \{ c+a : a\in \mathcal{A} \}$.
    \footnote{If $\mathcal{A}$ is bounded \& nonempty, then $sup (c+\mathcal{A}) = c + sup \mathcal{A}$.}
\end{notation}

\begin{definition}
    Set $\mathcal{A}$ is \uline{bounded above} \iff $\exist$ b $\in \R$ s.t, $\any a \in \R, a \leq b$.\\
    Similarly for \uline{bounded below}.
\end{definition}

\begin{definition}
    $s \in \R$ \uline{least upper bound} for Set $\mathcal{A} \subseteq \R$ \iff 

    \romannumeral 1 ) $s$ is on upper bound for $\mathcal{A}$
        
    \romannumeral 2 ) $\any b $ on upper bound for $\mathcal{A}, $ $s \leq b$\\
    where $s$ is noted as sup $\mathcal{A}$.
\end{definition}

\noindent \textbf{Axiom} \uwave{The axiom of completeness} (Least upper bound property)

\indent $\any set \mathcal{A} \subseteq \R, \mathcal{A}$ is non-empty, bounded above, $\exist sup \mathcal{A} $ for $\mathcal{A}$.

\begin{lemma}
    If $S \in \R$ is an upper bound for set $\mathcal{A} \subseteq \R$, Then\\
    \indent $S = sup \mathcal{A} \equiv \any \epsilon > 0$, $\exist a \in \mathcal{A}$ s.t, $S - \epsilon < a$.
\end{lemma}
\begin{proof}
    $(\Rightarrow)$ \\
    $S = sup \mathcal{A} \Rightarrow \any \epsilon > 0$, $\exist a \in \mathcal{A}$ such that $S - \epsilon < a$.\\
    $(\Leftarrow)$ \\
    Assume s $\neq sup \mathcal{A}, $ then take $\epsilon = \frac{1}{2} (S - sup \mathcal{A})$.
\end{proof}

\begin{theorem}[Nested Interval Theorem]
\label{Nested Interval Theorem}
$\any n \in \N$, $I_n := [a_n, b_n] \subseteq \R$. If $I_1 \supseteq I_2 \supseteq I_3 ... \supseteq I_n \supseteq I_{n+1}$, then $\bigcap\limits_{n=1}^{\infty} I_n \neq \emptyset$.
\end{theorem}

\begin{proof}
    $\mathcal{A} = \{a_n | n \in \N\}$\\
    since $\mathcal{A}$ is upper bounded, not empty, by \textit{C.A}., $\exist sup \mathcal{A} = x$.\\
    WTS: x $\in \bigcap\limits_{n=1}^{\infty} I_n$ \\
    Since x is sup $\mathcal{A}, a_n \leq x, \any n \in \N$.\\
    Since $b_n$ is upper bound for $\mathcal{A}$, and x = sup $\mathcal{A}$, x $\leq b_n$, $\any n \in \N$. \\
    thus, x $\in \bigcap\limits_{n=1}^{\infty} I_n \Rightarrow $ not  empty. \\
    * similarly, $inf \mathcal{B} \in \bigcap\limits_{n=1}^{\infty} I_n$.
\end{proof}

\begin{quote}
\begin{example}
$I_n = (0, \frac{1}{n})$, but $\bigcap\limits_{n=1}^{\infty} I_n = \emptyset$\\
\begin{proof}
    Assume not empty, then we can find x $\in (0, \frac{1}{n}), \any n \in \N$. \\
    Let n $\in \N$ be given, x $\in (0, \frac{1}{n}$).\\
    Then, $\exist \frac{1}{m} < x < \frac{1}{n}$, s.t, $x \notin (0, \frac{1}{m})$
\end{proof}
\end{example}
\end{quote}

\begin{theorem}[Density of \textbf{Q} in \textbf{R}]
\label{Density of Q in R}
$\any a, b \in \R,$ $\exist r \in \Q $ s.t, $a < r < b$. 
\footnote{Unfortunately, Archimedean Property for \N cannot be used for the proof.}
\end{theorem}

\begin{proof}
    Let m be the smallest integer, s.t, $m>na$. Then $m-1 \leq na<m \Rightarrow a < \frac{m}{n}$.\\
    Let n large enough, $\frac{1}{n} < b-a \equiv a < b- \frac{1}{n}$ 
\end{proof}

\section{Sequences}
\begin{definition}
    A \uline{sequence} is a \uwave{function} $a_n: \N \rightarrow \R$.
\end{definition}
\begin{definition}
    A sequence $\{a_n\}$ \uline{converges to a} $\in \R$ \iff $\any \epsilon > 0, \exist N \in \N$ s.t, $\any n \geq \N,$ $|a_n - a| < \epsilon \equiv a_n \in V_\epsilon (a)$. \\
    Noted as $\lim_{x\to\infty} a_n = a$, or $a_n \rightarrow$ a.
\end{definition}
\begin{definition}
    Given a $\in \R$, $\epsilon > 0, $ the set $V_\epsilon (a) = \{ x \in \R : |x-a|<\epsilon\}$ is called the \uline{$\epsilon$-neighborhood of $a$}.
\end{definition}
\begin{theorem}[Triangle Inequality] \hspace{1em}
\begin{enumerate}
    \item $|a+b| \leq |a| + |b|$. 
    \item $||a|-|b|| \leq |a-b|$.
\end{enumerate}
\end{theorem}
\begin{proof}
$|a|=|a-b+b|\leq |a-b| + |b| \Longrightarrow |a|-|b| \leq |a-b|$.\\
Similarly, $|b|-|a|\leq |a-b|$.\\
Then, $||a|-|b|| \leq |a-b|$.
\end{proof}
\begin{theorem}[Squeeze Theorem]
    If $x_n \leq y_n \leq z_n$, $\any n \in \N $, then
    
    $\lim_{n\to\infty} x_n = \lim_{n\to\infty} z_n = l \Rightarrow \lim_{n\to\infty} y_n = l$.
\end{theorem}
\begin{proof}
    (\uline{Order limit theorem} cannot be used since whether $\lim_{n\to\infty} y_n$ exists is unknown.)
    Let $\epsilon>0$ be given.
    \begin{enumerate}
        \item $\exist N_1 \in \N$ s,t. $\any n \in \N$, $n \geq N_1 \Rightarrow |x_n-l|<\epsilon$.
        \item $\exist N_2 \in \N$ s.t, $\any n \in \N$, $n \geq N_2 \Rightarrow |z_n-l|<\epsilon$.
    \end{enumerate}
    Take $N=max\{N_1, N_2\}$, then
    \[
        x_n \leq y_n \leq z_n \equiv -\epsilon < x_n - l \leq y_n -l \leq z_n - l <\epsilon \equiv |y_n - l| <\epsilon,
    \]
    for all $n\geq N$.
\end{proof}
\begin{theorem}[Monotone Converge Theorem]
    If $\{a_n\}$ is monotone and bounded, then $\{a_n\}$ converges.
\end{theorem}
\begin{proof}
    Since $\{a_n\}$ is bounded, by CA, $\{a_n\}$ has a sup$\{a_n\}=L$.\\
    WTS: $lim_{n\to\infty} a_n = L$.\\
    Let $\epsilon >0$ be given since L is the sup$\{a_n\}$, we know there is a $a_{n_0} \in \{a_n\}$ such that $a_{n_0} + \epsilon > L$. Take $N=n_0,$ since $a_n is \uparrow$, $\any n >N,$ we have, $L+\epsilon > L >a_n > a_{n_0} > L - \epsilon$.
\end{proof}

\section{Series}
\begin{definition}[Series]
    $\sum_{i=1}^{\infty} a_i = \lim_{n\to\infty} S_n = \lim_{n\to\infty} (a_1 + a_2 +...+a_n)$.
\end{definition}

\noindent \textbf{Often-used Skills}
Given $a_n \leq 0 \Rightarrow \{ S_n\} $ is monotonously increasing,
\begin{itemize}
    \item Prove: $\sum_{i=1}^{\infty} \frac{1}{n^2}$ converge. \\
    $S_n = \frac{1}{1\cdot 1} + \frac{1}{2\cdot2} +...+ \frac{1}{n\cdot n} < \frac{1}{1} + \frac{1}{1\cdot 2} + ...+ \frac{1}{(n-1)\cdot n} = 1 + 1 - \frac{1}{n} = 2 - \frac{1}{n}$
    \item Prove: 
    $\sum_{i=1}^{\infty} \frac{1}{n}$ diverge. \\
    $S_n = \frac{1}{1} + \frac{1}{2} + \frac{1}{3} + \frac{1}{4} +...+ \frac{1}{n}$. Let $n=2^k$\\
    $S_{2^k} = 1 + \frac{1}{2} + (\frac{1}{3}+\frac{1}{4})+...+\frac{1}{2^{k-1}}+\frac{1}{2^k}$\\
    > $1+\frac{1}{2}+(\frac{1}{2^2}+\frac{1}{2^2})+...+2^k \cdot \frac{1}{2^k}$\\
    = $1+\frac{1}{2}+\frac{1}{2}+...+\frac{1}{2} = 1+\frac{k}{2}$
\end{itemize}
\begin{definition}
    let $\{a_n\}$ be a sequence, $n_1 < n_2 <...$ be \uwave{increasing of natural numbers}.\\
    $\{b_n\}$ is a \uline{subsequence} of $\{a_n\}$ \iff $b_1 = a_{n_1}$, $b_2 = a_{n_2}, ... \any n \in \N$.
\end{definition}

\begin{theorem}
    subsequence of a convergent sequence converge to the same limit of the original sequence.
\end{theorem}
\begin{proof}
    Take $k\geq N$ s.t, $n_k \geq k \geq N$.
\end{proof}

\begin{example}
$\{x_n\} = \{1,-1, 1, -1...\}$ divergent.

\begin{proof}
$\{a_n\} = \{1,...,1\} \longrightarrow 1 \quad \& \quad \{b_n\} = \{-1,...,-1\} \longrightarrow -1$.\\
if $x_n \longrightarrow L$, then $a_n$, $b_n$  $\longrightarrow L$.
\end{proof}
\end{example}
\begin{theorem}[Bolzawo-Wereirstrass Theorem]
    If $\{a_n\}$ is bounded, then $\exist \{ a_{n_k} \} \subseteq \{ a_n\}$ is convergent.
\end{theorem}
% XXX
\begin{proof}
\begin{tikzpicture}[scale=1]
\coordinate (P) at (0, 0);
\coordinate (A) at (1, 0);
\coordinate (B) at (2, 0);
\coordinate (C) at (3.5, 0);
\coordinate (D) at (4, 0);
\coordinate (E) at (5, 0);
\coordinate (F) at (6, 0);
\coordinate (G) at (7, 0);
\draw[->] (P)--(G) node [below right]{$x$};
\draw (A) node[below] {$-M$};
\draw (C) node[below] {0};
\draw (F) node[below] {$M$} ;
\draw (A) node {$[$};
\draw (C) node {$][$};
\draw (F) node {$]$};
\draw (E) node[above] {$I_1$};
\end{tikzpicture}
\indent $\Longrightarrow$ \\
\begin{tikzpicture}[scale=1]
\coordinate (P) at (0, 0);
\coordinate (A) at (1, 0);
\coordinate (B) at (2, 0);
\coordinate (C) at (3.5, 0);
\coordinate (D) at (4, 0);
\coordinate (H) at (4.75, 0);
\coordinate (E) at (5, 0);
\coordinate (I) at (5.375, 0);
\coordinate (F) at (6, 0);
\coordinate (G) at (7, 0);
\draw[->] (P)--(G) node [below right]{$x$};
\draw (A) node[below] {$-M$} ;
\draw (C) node[below] {0} ;
\draw (F) node[below] {$M$}  ;
\draw (C) node {$[$};
\draw (H) node {$][$};
\draw (F) node {$]$};
\draw (I) node[above] {$I_2$};
\end{tikzpicture}
\indent $\cdots$\\
Thus, I have created nested intervals $I_1 \supseteq I_2 \supseteq \cdots \supseteq I_n \cdots$.\\
Given len($I_n$) $\rightarrow$ 0 as n $\rightarrow \infty$. Further, by Thm \ref{Nested Interval Theorem}, $\bigcap\limits_{i=1}^{\infty} I_i \neq \emptyset$.\\
Let x $\in \bigcap\limits_{i=1}^{\infty} I_i$ and $n_i \in \N$ s.t, $a_{n_i} \in I_i$\\
WTS: $\any \epsilon > 0$, $\exist k \in \N$ s.t, $\any k \in \N $, $k>K \Rightarrow |a_{n_k} - x| < \epsilon$\\
since len$(I_n) \rightarrow 0$, $\exist k$ s.t, len$(I_k)<\epsilon$.\\
Thus, $\any k \in \N$, $k>K, x a_{n_k} \in I_k \Rightarrow |a_{n_k} - x|<\epsilon$.
\end{proof}
\textbf{The Cauchy Criterion}
\begin{definition}
    A sequence $\{a_n\}$ is called a Cauchy sequence if $\any \epsilon >0$, $\exist N \in \N$ such that  $\any m,n \in \N$, $m,n \geq N \Rightarrow |a_n-a_m| < \epsilon$ (OR sup $x_i$ - inf $x_j \rightarrow 0$, $i \geq N$, $j \geq N$)
\end{definition}
\begin{lemma}
    Cauchy sequence is bounded.
\end{lemma}
\begin{proof}
    Let $\epsilon = 1$, $\exist N$ s.t, $|a_n-a_N| <1$, $\any n \geq N$\\
    Let M = max $\{a_1, a_2...a_n\}$\\
    Then $M' = M+1$ is an upper bound of $\{a_n\}$.
\end{proof}
\begin{theorem}
    $\{a_n\}$ is a Cauchy sequence $\equiv \{a_n\}$ convergent.
\end{theorem}
\begin{proof}
    ($\Rightarrow$) \\
    lemma + Bolzawo-Wereistrass thm. \\
    $\exist \{a_{n_k}\}$ convergent after $n_k$.
    Take $N_0 = max \{ n_k, N\}$, $\any n \in\N$, $|a_n - L| = |a_n - a_{n_k}| + |a_{n_k} - L| < \frac{\epsilon}{2} + \frac{\epsilon}{2} = \epsilon$. \\
    ($\Leftarrow$) \\
    $|a_n -a_m| < |a_n - L | + |a_m - L | = \frac{\epsilon}{2}+\frac{\epsilon}{2} = \epsilon$.
\end{proof}
