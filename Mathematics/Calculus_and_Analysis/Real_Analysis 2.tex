\begin{theorem}[Cauchy Criterion for Series]
\begin{align}
                          & \sum_{n=1}^\infty a_n \text{converges} \\
    \Leftrightarrow \quad & \any \epsilon>0\text{, }\exist N \in \N\text{ s.t }\any n \geq N\text{, }\left|\sum_{k=n+1}^\infty a_k \right| < \epsilon \\
    \Leftrightarrow \quad &\any \epsilon>0\text{, }\exist N \in \N\text{ s.t }\any n,m \geq N\text{, }\left|\sum_{k=n+1}^m a_k \right| < \epsilon
\end{align}
\end{theorem} 
\begin{proof}
    (1) $\Rightarrow$ (2)\qquad $ \left|L-S_n\right| = \left|\lim_{m\to\infty} S_m - S_n\right| = \left|\lim_{m\to\infty} \sum_{k=n+1}^m a_k\right|\\ = \left|\sum_{k=n+1}^\infty a_k\right| < \epsilon$. $\hfill\blacksquare$ \vspace{1em}\\
    (2) $\Rightarrow$ (3) \qquad $\left|\sum_{k=n+1}^\infty a_k\right| = \left|\sum_{k=n+1}^\infty a_k - \sum_{k=m+1}^\infty a_k\right| < \frac{\epsilon}{2} + \frac{\epsilon}{2} = \epsilon$.  $\hfill\blacksquare$ \vspace{1em}\\
    (3) $\Rightarrow$ (1) \qquad \{$S_n$\} is Cauchy sequence, thus $S_n$ converge. 
\end{proof}

\section*{Series Tests\footnote{
Let $\{a_n\}$ is bounded, $b_n=\sup_{k\geq n} a_k$. Then we define
\[
    \limsup a_n = \lim_{n \rightarrow \infty} \sup_{k\geq n} a_k.
\]
Similarly for $\liminf a_n$.
}}

\begin{theorem}[$n^{th}$ term test]
\label{n_th term test}
    If $\sum_{n=1}^\infty a_n$ converges, then $\lim_{n\to\infty} a_n = 0$.
\end{theorem}
\begin{proof}
    $\lim_{n\to\infty} a_n = \lim_{n\to\infty} (S_n -S_{n-1}) = L -L =0$
\end{proof}
\begin{example}
    (Counter-example) $a_n = \frac{1}{n}$.
\end{example}

\begin{theorem}[Comparison Test]
\label{Comparison Test}
     $|a_n| \leq b_n, \any n \geq 1$, 
 \[
 \begin{cases}
        \text{If $\sum_{n=1}^\infty b_n$ converges, then $\sum_{n=1}^\infty a_n$ converges and $\left|\sum_{n=1}^\infty a_n \right| < \sum_{n=1}^\infty$}. \\
        \text {If $\sum_{n=1}^\infty a_n$ diverges, then $\sum_{n=1}^\infty b_n$ diverges.}
    \end{cases}
\]    
\end{theorem}
\begin{proof}
    $\left|\sum_{k=n+1}^\infty a_n \right| \leq \sum_{k=n+1}^\infty |a_n| \leq \sum_{k=n+1}^\infty b_n$ since $\sum_{k=n+1}^\infty b_n$ is Cauchy, given $\epsilon > 0$, $\any n,m \in \N$, $\sum_{k=n+1}^\infty b_n <\epsilon$.
\end{proof}

\begin{theorem}[Root Test]
\label{Root Test}
    $a_n \geq 0$, $\any n \in \N$, $l = \lim$  sup$\sqrt[n]{a_n}$ ($a_n \equiv 1$).
\[    
    \begin{cases}
        \text{If } l <1 \text{, then} \sum_{n=1}^\infty a_n \text{ converges.} \\
        \text{If } l > 1 \text{, then} \sum_{n=1}^\infty a_n \text{diverges.}
    \end{cases}
\]
\end{theorem}
\begin{proof}
    Let $0<\epsilon<1-l$ be given. \\
    Since $l=\lim_{n\to\infty}$ sup$\sqrt[k]{a_k}$ ($k \geq n$), i.e. 
    \[
    \any \epsilon > 0, \exist N \in \N s.t \any n > N \Rightarrow |\sup_{k\geq n} a^{\frac{1}{k}}_k - l| < \epsilon,
    \]
    we have
    \[
    l-\epsilon < sup a_k^\frac{1}{k} < \epsilon + l < 1 , 1\geq a_n^\frac{1}{n},
    \]
    where $\sup_{k\geq n} a_k^{1/k} \geq a_k^{1/k}$ and $\epsilon + l < 1$. Thus,
    \[
    \sqrt[n]{a_n} \leq sup\sqrt[k]{a_k} < l + \epsilon <1, \any n \in \N, n >N.
    \]
    That is, $a_n \leq r^n$, $r=l + \epsilon < 1$.
\end{proof}

\begin{theorem}[Limit Comparison Test]
\label{Limit Comparison Test}
    If $b_n \geq 0$, \uwave{$\any n \in \N$} and $\lim$ sup$\frac{|a_n|}{b_n} < \infty$ and $\sum_{n=1}^\infty b_n$ converge, then $\sum_{n=1}^\infty$ converge.
\end{theorem}

\begin{theorem}[Integral Test]
\label{Integral Test}
    $f(x)>0$, monotone $\downarrow$ on $[x_0, \infty)$, $\sum_{n=1}^\infty f(n)$ converge $\Leftrightarrow \int_1^\infty f(x)$ converge.
\end{theorem}

\begin{theorem}[Ratio Test]
\label{Ratio Test}
    $a_n \geq0$, \uwave{$\any n \in \N$},
    $$\begin{cases}
        \text{If }\lim \text{sup} \frac{a_{n+1}}{a_n} <1 \text{, then } \sum_{n=1}^\infty a_n \text{ converge}.\\
        \text{If }\lim \text{sup} \frac{a_{n+1}}{a_n} >1 \text{, then } \sum_{n=1}^\infty a_n \text{ diverge}.
    \end{cases}$$
\end{theorem}

\begin{theorem}[Alternative Series Test]
\label{Alternative Series Test}
    If $a_n$ monotonously  $\downarrow$ and  $\lim_{n\to\infty} a_n =0$ then $\sum_{n=1}^\infty (-1)^n a_n$ converge. 
\end{theorem}
\begin{proof}
    Take $I_n = [S_{2n+2}, S_{2n+1}]$, it is nested. Thus, by Nested Interval Theorem \ref{Nested Interval Theorem}, $\bigcap\limits_{n=1}^{\infty} I_n \neq \emptyset$ and $len(I_n) = a_{2n+2} \rightarrow 0$.
\end{proof}
\begin{example}
    Taylor Series $\sum_{n=1}^\infty \cfrac{(-1)^{n-1}}{n} = \ln 2$.
\end{example}    


\section{Topology in $\R$}
\begin{definition}
    $V_\epsilon (X) = \{ a \in \R \bigl| |a-x| < \epsilon \} = (x-\epsilon, x+ \epsilon)$.
\end{definition}
\begin{definition}
    $I \in \R$ is \uline{open} if $\any x \in I$, $\exist \epsilon>0$ s.t $V_\epsilon (X) \subseteq I$.
\end{definition}

\begin{theorem} \hspace{1em}
\begin{enumerate}
    \item Any union of open sets is open.
    \item Any \uwave{finite} intersection of open sets is open.
\end{enumerate}
\end{theorem}
\begin{proof}
    (1) Let $I_\alpha$ be open, $\any \alpha \in A$, $I = \bigcup\limits_{\alpha \in A} I_\alpha$, $x \in I \Rightarrow x \in I_\alpha$, for some $\alpha \in A$. \\
    Thus, $\exist \epsilon >0$ s.t $V_\epsilon (x) \subseteq I_\alpha \subseteq I$. $\hfill\blacksquare$ \\
    (2) Let $I_\alpha$ be open, $\alpha \in \{1,2,...,n\}$, let $\epsilon = min\{ \epsilon_1,...,\epsilon_n\}$, we have $V_\epsilon (x) \subseteq V_{\epsilon i} (x)\subseteq I_\alpha$, $\any i \Rightarrow V_\epsilon (X) \subseteq I$.
\end{proof}

\begin{definition}
    x is a \uline{limit point of A} if $\any \epsilon > 0$, $V_\epsilon(\epsilon \cap A) \setminus \{x\} \neq \emptyset$, $a_n \neq x$.
\end{definition}
\begin{theorem}
    x is a LP of $\mathcal{A}$ $\Leftrightarrow \exist \{ a_n \}_{n=1}^\infty \subseteq \mathcal{A}$ s.t $a_n \neq x$, $\any n \in \N$, and $\lim_{n\to\infty} a_n = x$.
\end{theorem}
\begin{proof}
    ($\Rightarrow$) \\
    Let $\epsilon_n = \frac{1}{n}$, and form a sequence {$a_n$} by picking $a_n \in V_{\epsilon}(x)\cap\mathcal{A}$. \\
    ($\Leftarrow$) \\
    By definition, $\any \epsilon > 0$, we have $V_\epsilon(\epsilon_n \cap A) \setminus \{x\} \neq \emptyset$.
\end{proof}
\begin{definition}
    $A \subseteq \R$ is closed if $\any x$ be a LP of A, $x \in A$.
\end{definition}
\begin{definition}
    x is an \uline{isolated point} of A if it is not a L.P. of A.
\end{definition}
\begin{theorem}
    $\any x \in \R$, $\exist \{a_n\} \subseteq \Q$ s.t $\lim_{n\to\infty} a_n = x$ ("$\Q$ is dense in $\R$").
\end{theorem}
\begin{proof}
    $\any u,v \in \R$, $u<v$, $\exist\epsilon \in \Q$, $u<\epsilon<v$ \\
    $\Rightarrow \any n\in \N$, $\exist q_n \in \Q$, $x-\frac{1}{n} < q_n <x+ \frac{1}{n}$\\
    $\Rightarrow |x-\epsilon_n| < \frac{2}{n} \rightarrow 0 \Rightarrow \lim_{n\to\infty} a_n =x$.
\end{proof}
\begin{definition}
    A set $K \subseteq \R$ is \uline{compact} if $\any \{a_n\} \subseteq k$, $\exist \{ a_{n_k} \} \subseteq\{a_n\}$ s.t $\lim_{k\to\infty} a_{n_k} = L \in K$.
\end{definition}
\begin{theorem}
    A set $K$ is compact $\Leftrightarrow$ K is closed and bounded.
\end{theorem}
\begin{proof}
    ($\Rightarrow$)\\
    Assume $K$ is not bounded, then $\exist x_n \in K$ s.t, $x_n \notin (-n,n)$. Thus \{$x_n$\} diverge. \\
    Assume $K$ is not closed, then $\exist x \notin K$ but a L.P. of $K$.Thus $\lim x_n = x$. Take the subsequence of it. \\
    ($\Leftarrow$) \\
    Exercise.
\end{proof}

\begin{theorem}[Nested Compact Set Property]\footnote{The difference between Nested Interval Theorem \ref{Nested Interval Theorem} is that $K$ could be a non-interval.}
\label{Nested Compact Set Property} 
    If $K_1 \supseteq K_2 \supseteq...$ is a nested sequence, $\any i \in \N$, $K_i$ is compact and non-empty, then $\bigcap\limits_{i=1}^\infty K_i \neq \emptyset$.
\end{theorem}
\begin{proof}
    Pick $x_1 \in K_1$, $x_2 \in K_2 ...$ then, we obtain a sequence $\{ x_n\}$ of pts, and $\any i \in \N$, $\{x_n\}_{n=i}^\infty \subseteq K_i$ \\
    Since $K_1$ is compact, $\{x_n\} \subseteq K_1$, $\exist\{x_{n_k}\}$ s.t $\lim_{k\to\infty} x_{n_k} = x \in K_1$. Elimary some first few terms, $\{x_{n_k}\}_{k=j}^\infty \in K_i$, $\any i \in \N$. \\
    Since $\{x_{n_k}\}_{k=j}^\infty \in K_i \rightarrow x$, subsequence of it $\rightarrow x \in K_i$, $\any i \in \N$.
\end{proof}
\begin{definition}
    $A \in \R$, an \underline{open cover} for A is a (possibly \uwave{infinite}) collection of open sets $\{a_\lambda | \lambda \in \Lambda\}$ s.t $A \subseteq U_{\lambda \in \Lambda} O_\lambda$.
\end{definition}
\begin{definition}
    Given an open cover for A, a \underline{finite subcover} is a finite collection of open sets from the original open cover whose union still completely contains A.
\end{definition}
\begin{theorem}[Heine-Borel Theorem]
\label{Heine-Borel Theorem}
    Let $K \subseteq \R$, then: \\
    \indent K is compact \\
    $\Leftrightarrow$ K is closed and bounded \\
    $\Leftrightarrow \any$ an open cover $O_c$ of k, $O_c$ has a finite subcover.
\end{theorem}
% XXX

\begin{theorem}
    $E \subseteq \R$ is connected  $\Leftrightarrow$ $\any A,B \neq \emptyset$, $A \cap B = \emptyset$, if $E = A \cup B$, then $\exist \{ x_n \} \in A$ (or B) s.t $\{ x_n \} \rightarrow x \in B$ (or A).\\
    \indent \indent \indent \indent \indent \indent \indent \indent$\Leftrightarrow$ $\any a,b,c \in \R$, $a<c<b$, if $a,b \in E$, then $c \in E$.
\end{theorem}
\begin{proof}
    (i) $\Rightarrow$ (iii) \\
    Let $A =  (-\infty, C) \cap E \ni a$, $B = (C, +\infty) \cap E \ni b$, and A, B separated. Since E is connected, E $\neq A \cup B = R \cap E \setminus \{C\} \Rightarrow C \in E$. $\hfill\blacksquare$

\begin{tikzpicture}[scale=1]
\coordinate (P) at (0, 0);
\coordinate (A) at (1, 0);
\coordinate (B) at (2, 0);
\coordinate (C) at (3, 0);
\coordinate (D) at (4, 0);
\coordinate (E) at (5, 0);
\draw[->] (P)--(E) node [below right]{$x$};
\draw (A) node[above] {$A$};
\draw (B) node[below] {E};
\draw (C) node[above] {$B$} ;
\draw (P) node {$($};
\draw (B) node {$)($};
\draw (D) node {$)$};
    \end{tikzpicture}\\
    (iii) $\Rightarrow$ (ii)\\
    Let $A, B \neq \emptyset$, $A \cap B = \emptyset$, $E = A \cup B$. \\
    Let $a_0 \in A$, $b_0 \in B$, then $a_0<c_0<b_0$. \\
    Let $a_1 \in A$, $a_1>a_0$, $b_1 \in B$, $b_1<b_0$, choose $c_1$ divide $[a_1,b_1]$ into half. Thus, we get
    \[
        I_1 \supseteq I_2 \supseteq...\supseteq I_n \supseteq...
    \]
    Then, $\bigcap\limits_{i=1}^\infty I_i \neq \emptyset$, since $len(I_n) \rightarrow 0, \existonly c \in \bigcap\limits_{i=1}^\infty I_i$ and $c_n \rightarrow c \in E$ (not enough). We have 
    \begin{enumerate}
        \item Suppose $c \in A$ then $\{b_n\} \subseteq B$, $b_n \rightarrow c$;
        \item Suppose $c \in B$ then $\{a_n\} \subseteq A$, $a_n \rightarrow c$.$\hfill\blacksquare$
    \end{enumerate} 
    Either $A\cap B = \emptyset$ or $B \cap A = \emptyset \Rightarrow$ E must be connected.
\end{proof}
\begin{definition}
\label{Cantor set}
    Cantor set $[ G = \bigcap\limits_{n=1}^\infty S_n]$ \\
    $S_0 = [0,1]$\\
    $S_1 = [0,\frac{1}{3}]\cup [\frac{2}{3},1]$\\
    $S_2 = [0, \frac{1}{9}] \cup [\frac{2}{9}, \frac{1}{3}]\cup [\frac{2}{3}, \frac{7}{9}]\cup[\frac{8}{9}, 1]$\\
    $\vdots$ \\
    $\Rightarrow$ (1)  $\bigcap\limits_{n=1}^\infty S_n$ is compact. (2) $\bigcap\limits_{n=1}^\infty S_n$ measure zero. (3) $\#\bigcap\limits_{n=1}^\infty S_n = \infty$ uncountable. (4) Int $\bigcap\limits_{n=1}^\infty S_n = \emptyset$ (Int = (open) interval (e.x. Int $[a,b] =(a,b)$). %反证 
    (5) G is a \uwave{perfect set}.
    %XXX
\end{definition}
\begin{proof}
    (5) Let $x \in G$ \\
    \textcircled{1} x is not the right end point of one of the interval of $S_n$. Then, let $x_n$ be the right end point of $I_n \subseteq S_n$, $x \in I_n$, then we have $x_n \rightarrow x$. \\
    \textcircled{2} x is the right end point, then take $x_n$ be the left end pts.
\end{proof}
\begin{proof}
    (4) Removing $=\frac{1}{3} + \frac{2}{3 \times 3} + ... = \sum_{n=0}^\infty \frac{2^n}{3^{n+1}} = \frac{1}{3} \cdot \frac{1}{1-\frac{2}{3}} = 1$
    $\therefore \|G\| =\|[0,1]\| - removing = 1-1 =0$
\end{proof}
\begin{theorem}
    A non-empty perfect set is uncountable.
\end{theorem}
% page 8
\begin{definition}
    $f:A\rightarrow \R$ is a function. c is a \uwave{limit point} of the demain A. Then $\lim_{x\to\ c} f(x) = L \Leftrightarrow \any \epsilon >0$, $\exist \delta >0$ s.t $0<|x-c|<\delta \Rightarrow |f(x)-L| < \epsilon$
\end{definition}
\begin{theorem}[Sequential Criterion for Functional Limits]
\label{Sequential Criterion for Functional Limits}
    Given $f:A\rightarrow \R$, c is a limit point of A.\\
    (i) $\lim_{x\to\ c} f(x) = L\\ \Leftrightarrow$ 
    (ii) $\any \{x_n\} \subseteq A$ s.t $x_n \neq c$ and $\lim_{n\to\infty} x_n = c \Rightarrow \lim_{n\to\infty} f(x_n) = L.$
\end{theorem}
\begin{proof}
    ($\Rightarrow$) \\
    Let $\{x_n\} \subseteq A$ be given. $x_n \neq c$, $\epsilon > 0$ be given. \\
    Since $\lim_{x\to\ c} f(x) = c$, $\exist \delta$ s.t $0<|x-c|<\delta_0 \Rightarrow |f(x) -L| <\epsilon$. \\
    Since $\lim_{x\to\infty} x_n = c$, $\exist N $ s.t $n>N \Rightarrow 0<|x_n -c| <\delta$ for the $\delta$. \\
    Take $N = N_0$, assume $n>N$, then:\\
    $0<|x_n-c|<\delta \Rightarrow |f(x_n) -L| <\epsilon$\\
    ($\Leftarrow$) \\
    Assume $\exist \epsilon > 0$ s.t $\any \delta >0$, $\exist x_0 \in A$, $0<|x_0 -c| < \delta$ and $|f(x)-L| \geq \epsilon$.
    Let $\delta = \frac{1}{n}$, $\exist x_n \neq c$ s.t $0<|x_n-c|<\frac{1}{n}$ and $|f(x_n) -L| \geq \epsilon$, $\any n \in \N$, thus, we make a sequence $\{x_n\}$,\\
    \uwave{thus, $x_n \rightarrow c$} $\Rightarrow |f(x_n) - L| <\epsilon$.
\end{proof}
\begin{theorem}[Algebraic Limit Theorem]
\label{Algebraic Limit Theorem}
    $f:A \rightarrow \R$, c is a limit point on A. \\
    $\exist \{ x_n\}, \{y_n\} \subseteq A$, $x_n \neq c$, $y_n \neq c$, $\lim_{n\to\infty} x_n = \lim_{n\to\infty} y_n = c$, $\lim_{n\to\infty} f(x_n) \neq \lim_{x\to\infty} f(y_n) \Rightarrow \lim_{x\to\ c} f(x)$ D.N.E.
\end{theorem}
\begin{example}
    $$
        f(x) = 
    \begin{cases}
        \text{1 } x \in \Q \\
        \text{0 } x \in \Q^c
    \end{cases}$$ \\
    Let $\delta = \frac{1}{n}$, then take $x_n \in \Q \cap V_\delta (a)$, $y_n \in \Q^c \cap V_\delta (a)$, then $x_n \rightarrow a$, $y_n \rightarrow a$, $\any a \in \R$, but $f(x_n) =1$, $f(y_n) = 0$\\
    $\therefore \lim_{x\to\ a} f(x)$ D.N.E., $\any a \in \R$.
\end{example}
\begin{definition}
    $f:A\rightarrow\R$ is continuous at a point of A, $c \in A$ if $\any \epsilon > 0$, $\exist \delta >0$ s.t $\uwave{|x-c|<\delta }\Rightarrow |f(x)-f(c)| < \epsilon$. %\therefore 不需要 limit points
\end{definition}
\begin{definition}
    $f:A\rightarrow\R$ is continuous at a point of A if f is continuous at every point of A.
\end{definition}
\uwave{Criterion of Discontinuity} \\
Given $f: A\rightarrow \R$, $c \in A$ be a limit point of A. \\
\indent $\exist \{x_n\} \subseteq A$ s.t $x_n \rightarrow c$ but $f(x_n) \nrightarrow f(c)$. \\
$\Rightarrow$ f is not c.t.s at c.
