\begin{theorem}[Cauchy Criterion for Series]
\begin{align}
                          & \sum_{n=1}^\infty a_n \text{converges} \\
    \Leftrightarrow \quad & \any \epsilon>0\text{, }\exist N \in \N\text{ s.t }\any n \geq N\text{, }\left|\sum_{k=n+1}^\infty a_k \right| < \epsilon \\
    \Leftrightarrow \quad &\any \epsilon>0\text{, }\exist N \in \N\text{ s.t }\any n,m \geq N\text{, }\left|\sum_{k=n+1}^m a_k \right| < \epsilon
\end{align}
\end{theorem} 
\begin{proof}
    (1) $\Rightarrow$ (2)\qquad $ \left|L-S_n\right| = \left|\lim_{m\to\infty} S_m - S_n\right| = \left|\lim_{m\to\infty} \sum_{k=n+1}^m a_k\right|\\ = \left|\sum_{k=n+1}^\infty a_k\right| < \epsilon$. $\hfill\blacksquare$ \vspace{1em}\\
    (2) $\Rightarrow$ (3) \qquad $\left|\sum_{k=n+1}^\infty a_k\right| = \left|\sum_{k=n+1}^\infty a_k - \sum_{k=m+1}^\infty a_k\right| < \cfrac{\epsilon}{2} + \cfrac{\epsilon}{2} = \epsilon$.  $\hfill\blacksquare$ \vspace{1em}\\
    (3) $\Rightarrow$ (1) \qquad \{$S_n$\} is Cauchy sequence, thus $S_n$ converge. 
\end{proof}

\section*{Series Tests\footnote{
Let $\{a_n\}$ is bounded, $b_n=\sup_{k\geq n} a_k$. Then we define
\[
    \limsup a_n = \lim_{n \rightarrow \infty} \sup_{k\geq n} a_k.
\]
Similarly for $\liminf a_n$.
}}

\begin{theorem}[$n^{th}$ term test]
\label{n_th term test}
    If $\sum_{n=1}^\infty a_n$ converges, then $\lim_{n\to\infty} a_n = 0$.
\end{theorem}
\begin{proof}
    $\lim_{n\to\infty} a_n = \lim_{n\to\infty} (S_n -S_{n-1}) = L -L =0$
\end{proof}
\begin{example}
    (Counter-example) $a_n = \cfrac{1}{n}$.
\end{example}

\begin{theorem}[Comparison Test]
\label{Comparison Test}
     $|a_n| \leq b_n, \any n \geq 1$, 
 \[
 \begin{cases}
        \text{If $\sum_{n=1}^\infty b_n$ converges, then $\sum_{n=1}^\infty a_n$ converges and $\left|\sum_{n=1}^\infty a_n \right| < \sum_{n=1}^\infty$}. \\
        \text {If $\sum_{n=1}^\infty a_n$ diverges, then $\sum_{n=1}^\infty b_n$ diverges.}
    \end{cases}
\]    
\end{theorem}
\begin{proof}
    $\left|\sum_{k=n+1}^\infty a_n \right| \leq \sum_{k=n+1}^\infty |a_n| \leq \sum_{k=n+1}^\infty b_n$ since $\sum_{k=n+1}^\infty b_n$ is Cauchy, given $\epsilon > 0$, $\any n,m \in \N$, $\sum_{k=n+1}^\infty b_n <\epsilon$.
\end{proof}

\begin{theorem}[Root Test]
\label{Root Test}
    $a_n \geq 0$, $\any n \in \N$, $l = \limsup \sqrt[n]{a_n}$ ($a_n \equiv 1$).
\[    
    \begin{cases}
        \text{If } l <1 \text{, then} \sum_{n=1}^\infty a_n \text{ converges.} \\
        \text{If } l > 1 \text{, then} \sum_{n=1}^\infty a_n \text{diverges.}
    \end{cases}
\]
\end{theorem}
\begin{proof}
    Let $0<\epsilon<1-l$ be given. \\
    Since $l=\lim_{n\to\infty} \sup_{k \geq n} \sqrt[k]{a_k}$, i.e. 
    \[
    \any \epsilon > 0, \exist N \in \N s.t \any n > N \Rightarrow |\sup_{k\geq n} a^{1/k}_k - l| < \epsilon,
    \]
    we have
    \[
    l-\epsilon < \sup a_k^{1/k} < \epsilon + l < 1 , 1\geq a_n^{1/n},
    \]
    where $\sup_{k\geq n} a_k^{1/k} \geq a_k^{1/k}$ and $\epsilon + l < 1$. Thus,
    \[
    \sqrt[n]{a_n} \leq \sup\sqrt[k]{a_k} < l + \epsilon <1, \any n \in \N, n >N.
    \]
    That is, $a_n \leq r^n$, $r=l + \epsilon < 1$.
\end{proof}

\begin{theorem}[Limit Comparison Test]
\label{Limit Comparison Test}
    If $b_n \geq 0$, \uwave{$\any n \in \N$} and $\lim \sup \frac{|a_n|}{b_n} < \infty$ and $\sum_{n=1}^\infty b_n$ converge, then $\sum_{n=1}^\infty$ converge.
\end{theorem}

\begin{theorem}[Integral Test]
\label{Integral Test}
    $f(x)>0$, monotone $\downarrow$ on $[x_0, \infty)$, $\sum_{n=1}^\infty f(n)$ converge $\Leftrightarrow \int_1^\infty f(x)$ converge.
\end{theorem}

\begin{theorem}[Ratio Test]
\label{Ratio Test}
    $a_n \geq0$, \uwave{$\any n \in \N$},
    \[\begin{cases}
        \text{If }\limsup \cfrac{a_{n+1}}{a_n} <1 \text{, then } \sum_{n=1}^\infty a_n \text{ converge}.\\
        \text{If }\limsup \cfrac{a_{n+1}}{a_n} >1 \text{, then } \sum_{n=1}^\infty a_n \text{ diverge}.
    \end{cases}\]
\end{theorem}

\begin{theorem}[Alternative Series Test]
\label{Alternative Series Test}
    If $a_n$ monotonously  $\downarrow$ and  $\lim_{n\to\infty} a_n =0$ then $\sum_{n=1}^\infty (-1)^n a_n$ converge. 
\end{theorem}
\begin{proof}
    Take $I_n = [S_{2n+2}, S_{2n+1}]$, it is nested. Thus, by Nested Interval Theorem \ref{Nested Interval Theorem}, $\bigcap\limits_{n=1}^{\infty} I_n \neq \emptyset$ and $len(I_n) = a_{2n+2} \rightarrow 0$.
\end{proof}
\begin{example}
    Taylor Series $\sum_{n=1}^\infty \cfrac{(-1)^{n-1}}{n} = \ln 2$.
\end{example}    


\section{Topology in $\R$}
\begin{definition}
    $\mathcal{V}_\epsilon (x) = \{ a \in \R \bigl| |a-x| < \epsilon \} = (x-\epsilon, x+ \epsilon)$.
\end{definition}
\begin{definition}
    $I \in \R$ is \uline{open} if $\any x \in I$, $\exist \epsilon>0$ s.t $\mathcal{V}_\epsilon (x) \subseteq I$.
\end{definition}

\begin{theorem} \hspace{1em}
\begin{enumerate}
    \item Any union of open sets is open.
    \item Any \uwave{finite} intersection of open sets is open.
\end{enumerate}
\end{theorem}
\begin{proof}
    (1) Let $I_\alpha$ be open, 
    \[
    \any \alpha \in \mathcal{A}, I = \bigcup\limits_{\alpha \in A} I_\alpha, x \in I \Rightarrow x \in I_\alpha, \text{for some } \alpha \in \mathcal{A}.
    \] \\
    Thus, $\exist \epsilon >0$ s.t $\mathcal{V}_\epsilon (x) \subseteq I_\alpha \subseteq I$. $\hfill\blacksquare$ \\
    (2) Let $I_\alpha$ be open, $\alpha \in \{1,2,...,n\}$, let $\epsilon = min\{ \epsilon_1,...,\epsilon_n\}$, we have 
    \[
    \mathcal{V}_\epsilon (x) \subseteq \mathcal{V}_{\epsilon i} (x)\subseteq I_\alpha, \any i \Rightarrow \mathcal{V}_\epsilon (x) \subseteq I.
    \]
\end{proof}

\begin{definition}
    x is a \uline{limit point of $\mathcal{A}$} if $\any \epsilon > 0$, $V_\epsilon(\epsilon \cap \mathcal{A}) \setminus \{x\} \neq \emptyset$, $a_n \neq x$.
\end{definition}
\begin{theorem}
    x is a L.P. of $\mathcal{A}$ $\Leftrightarrow \exist \{ a_n \}_{n=1}^\infty \subseteq \mathcal{A}$ s.t $a_n \neq x$, $\any n \in \N$, and $\lim_{n\to\infty} a_n = x$.
\end{theorem}
\begin{proof}
    ($\Rightarrow$) \\
    Let $\epsilon_n = \cfrac{1}{n}$, and form a sequence {$a_n$} by picking $a_n \in V_{\epsilon}(x)\cap\mathcal{A}$. \\
    ($\Leftarrow$) \\
    By definition, $\any \epsilon > 0$, we have $V_\epsilon(\epsilon_n \cap \mathcal{A}) \setminus \{x\} \neq \emptyset$.
\end{proof}
\begin{definition}
    $\mathcal{A} \subseteq \R$ is closed if $\any x$ be a L.P. of $\mathcal{A}$, $x \in \mathcal{A}$.
\end{definition}
\begin{definition}
    x is an \uline{isolated point} of $\mathcal{A}$ if it is not a L.P. of $\mathcal{A}$.
\end{definition}
\begin{theorem}
    $\any x \in \R$, $\exist \{a_n\} \subseteq \Q$ s.t $\lim_{n\to\infty} a_n = x$ ("$\Q$ is dense in $\R$").
\end{theorem}
\begin{proof}
    $\any u,v \in \R$, $u<v$, $\exist\epsilon \in \Q$, $u<\epsilon<v$ \\
    $\Rightarrow \any n\in \N$, $\exist q_n \in \Q$, $x-\cfrac{1}{n} < q_n <x+ \cfrac{1}{n}$\\
    $\Rightarrow |x-\epsilon_n| < \cfrac{2}{n} \rightarrow 0 \Rightarrow \lim_{n\to\infty} a_n =x$.
\end{proof}
\begin{definition}
    A set $\mathcal{K} \subseteq \R$ is \uline{compact} if $\any \{a_n\} \subseteq \mathcal{K}$, $\exist \{ a_{n_k} \} \subseteq\{a_n\}$ s.t $\lim_{k\to\infty} a_{n_k} = L \in \mathcal{K}$.
\end{definition}
\begin{theorem}
    A set $\mathcal{K}$ is compact $\Leftrightarrow$ $\mathcal{K}$ is closed and bounded.
\end{theorem}
\begin{proof}
    ($\Rightarrow$)\\
    Assume $\mathcal{K}$ is not bounded, then $\exist x_n \in \mathcal{K}$ s.t, $x_n \notin (-n,n)$. Thus \{$x_n$\} diverge. \\
    Assume $\mathcal{K}$ is not closed, then $\exist x \notin \mathcal{K}$ but a L.P. of $\mathcal{K}$.Thus $\lim x_n = x$. Take the subsequence of it. \\
    ($\Leftarrow$) \\
    Exercise.
\end{proof}

\begin{theorem}[Nested Compact Set Property]\footnote{The difference between Nested Interval Theorem \ref{Nested Interval Theorem} is that $K$ could be a non-interval.}
\label{Nested Compact Set Property} 
    If $\mathcal{K}_1 \supseteq \mathcal{K}_2 \supseteq...$ is a nested sequence, $\any i \in \N$, $\mathcal{K}_i$ is compact and non-empty, then $\bigcap\limits_{i=1}^\infty \mathcal{K}_i \neq \emptyset$.
\end{theorem}
\begin{proof}
    Pick $x_1 \in \mathcal{K}_1$, $x_2 \in \mathcal{K}_2 ...$ then, we obtain a sequence $\{ x_n\}$ of pts, and $\any i \in \N$, $\{x_n\}_{n=i}^\infty \subseteq \mathcal{K}_i$. \\
    Since $\mathcal{K}_1$ is compact, $\{x_n\} \subseteq \mathcal{K}_1$, $\exist\{x_{n_k}\}$ s.t $\lim_{k\to\infty} x_{n_k} = x \in \mathcal{K}_1$. Elimary some first few terms, $\{x_{n_k}\}_{k=j}^\infty \in \mathcal{K}_i$, $\any i \in \N$. \\
    Since $\{x_{n_k}\}_{k=j}^\infty \in \mathcal{K}_i \rightarrow x$, subsequence of it $\rightarrow x \in \mathcal{K}_i$, $\any i \in \N$.
\end{proof}
\begin{definition}
    $\mathcal{A} \subseteq \R$, an \uline{open cover} for $\mathcal{A}$ is a (possibly \uwave{infinite}) collection of open sets $\{a_\lambda | \lambda \in \Lambda\}$ s.t $\mathcal{A} \subseteq U_{\lambda \in \Lambda} O_\lambda$.
\end{definition}
\begin{definition}
    Given an open cover for $\mathcal{A}$, a \uline{finite subcover} is a finite collection of open sets from the original open cover whose union still completely contains $\mathcal{A}$.
\end{definition}
\begin{theorem}[Heine-Borel Theorem]
\label{Heine-Borel Theorem}
    Let $\mathcal{K} \subseteq \R$, then: \\
    \indent $\mathcal{K}$ is compact \\
    $\Leftrightarrow$ $\mathcal{K}$ is closed and bounded \\
    $\Leftrightarrow \any$ an open cover $O_c$ of $\mathcal{K}$, $O_c$ has a finite subcover.
\end{theorem}
% XXX

\begin{theorem}
    $\mathcal{E} \subseteq \R$ is connected  $\Leftrightarrow$ $\any \mathcal{A,B} \neq \emptyset$, $\mathcal{A} \cap \mathcal{B} = \emptyset$, if $\mathcal{E} = \mathcal{A} \cup \mathcal{B}$, then $\exist \{ x_n \} \in \mathcal{A}$ (or $\mathcal{B}$) s.t $\{ x_n \} \rightarrow x \in \mathcal{B}$ (or $\mathcal{A}$).\\
    \indent \indent \indent \indent \indent \indent \indent \indent$\Leftrightarrow$ $\any a,b,c \in \R$, $a<c<b$, if $a,b \in \mathcal{E}$, then $c \in \mathcal{E}$.
\end{theorem}
\begin{proof}
    (i) $\Rightarrow$ (iii) \\
    Let $\mathcal{A} =  (-\infty, \mathcal{C}) \cap \mathcal{E} \ni a$, $\mathcal{B} = (\mathcal{C}, +\infty) \cap \mathcal{E} \ni b$, and $\mathcal{A, B}$ separated. Since $\mathcal{E}$ is connected, 
    \[
    \mathcal{E} \neq \mathcal{A} \cup \mathcal{B} = \mathcal{R} \cap \mathcal{E} \setminus \{\mathcal{C}\} \Rightarrow \mathcal{C} \in \mathcal{E}.
    \]. $\hfill\blacksquare$

\begin{tikzpicture}[scale=1]
\coordinate (P) at (0, 0);
\coordinate (A) at (1, 0);
\coordinate (B) at (2, 0);
\coordinate (C) at (3, 0);
\coordinate (D) at (4, 0);
\coordinate (E) at (5, 0);
\draw[->] (P)--(E) node [below right]{$x$};
\draw (A) node[above] {$\mathcal{A}$};
\draw (B) node[below] {E};
\draw (C) node[above] {$\mathcal{B}$} ;
\draw (P) node {$($};
\draw (B) node {$)($};
\draw (D) node {$)$};
    \end{tikzpicture}\\
    (iii) $\Rightarrow$ (ii)\\
    Let $\mathcal{A, B} \neq \emptyset$, $\mathcal{A} \cap \mathcal{B} = \emptyset$, $\mathcal{E} = \mathcal{A} \cup \mathcal{B}$. \\
    Let $a_0 \in \mathcal{A}$, $b_0 \in \mathcal{B}$, then $a_0<c_0<b_0$. \\
    Let $a_1 \in \mathcal{A}$, $a_1>a_0$, $b_1 \in \mathcal{B}$, $b_1<b_0$, choose $c_1$ divide $[a_1,b_1]$ into half. Thus, we get
    \[
        I_1 \supseteq I_2 \supseteq...\supseteq I_n \supseteq...
    \]
    Then, $\bigcap\limits_{i=1}^\infty I_i \neq \emptyset$, since $len(I_n) \rightarrow 0, \existonly c \in \bigcap\limits_{i=1}^\infty I_i$ and $c_n \rightarrow c \in \mathcal{E}$ (not enough). We have 
    \begin{enumerate}
        \item Suppose $c \in \mathcal{A}$ then $\{b_n\} \subseteq \mathcal{B}$, $b_n \rightarrow c$;
        \item Suppose $c \in \mathcal{B}$ then $\{a_n\} \subseteq \mathcal{A}$, $a_n \rightarrow c$.$\hfill\blacksquare$
    \end{enumerate} 
    Either $\mathcal{A}\cap \mathcal{B} = \emptyset$ or $\mathcal{B} \cap \mathcal{A} = \emptyset \Rightarrow$ $\mathcal{E}$ must be connected.
\end{proof}
\begin{definition}
\label{Cantor set}
    Cantor set $[ \mathcal{G} = \bigcap\limits_{n=1}^\infty \mathcal{S}_n]$ \\
    $\mathcal{S}_0 = [0,1]$\\
    $\mathcal{S}_1 = [0,\cfrac{1}{3}]\cup [\cfrac{2}{3},1]$\\
    $\mathcal{S}_2 = [0, \cfrac{1}{9}] \cup [\cfrac{2}{9}, \cfrac{1}{3}]\cup [\cfrac{2}{3}, \cfrac{7}{9}]\cup[\cfrac{8}{9}, 1]$\\
    $\vdots$ \\
    $\Rightarrow$ (1)  $\bigcap\limits_{n=1}^\infty \mathcal{S}_n$ is compact. (2) $\bigcap\limits_{n=1}^\infty \mathcal{S}_n$ measure zero. (3) $\#\bigcap\limits_{n=1}^\infty \mathcal{S}_n = \infty$ uncountable. (4) Int $\bigcap\limits_{n=1}^\infty \mathcal{S}_n = \emptyset$ (Int = (open) interval (e.x. Int $[a,b] =(a,b)$). %反证 
    (5) $\mathcal{G}$ is a \uwave{perfect set}.
    %XXX
\end{definition}
\begin{proof}
    (5) Let $x \in \mathcal{G}.$ \\
    \textcircled{1} x is not the right end point of one of the interval of $\mathcal{S}_n$. Then, let $x_n$ be the right end point of $I_n \subseteq \mathcal{S}_n$, $x \in I_n$, then we have $x_n \rightarrow x$. \\
    \textcircled{2} x is the right end point, then take $x_n$ be the left end pts.
\end{proof}
\begin{proof}
    (4) Removing $=\cfrac{1}{3} + \cfrac{2}{3 \times 3} + ... = \sum_{n=0}^\infty \cfrac{2^n}{3^{n+1}} = \cfrac{1}{3} \cdot \cfrac{1}{1-\frac{2}{3}} = 1$
    $\therefore \|\mathcal{G}\| =\|[0,1]\| - removing = 1-1 =0$.
\end{proof}
\begin{theorem}
    A non-empty perfect set is uncountable.
\end{theorem}
% page 8
\begin{definition}
    $f:\mathcal{A}\rightarrow \R$ is a function. c is a \uwave{limit point} of the domain $\mathcal{A}$. Then \[
    \lim_{x\to\ c} f(x) = L \Leftrightarrow \any \epsilon >0, \exist \delta >0 s.t 0<|x-c|<\delta \Rightarrow |f(x)-L| < \epsilon.
    \]
\end{definition}
\begin{theorem}[Sequential Criterion for Functional Limits]
\label{Sequential Criterion for Functional Limits}
    Given $f:\mathcal{A}\rightarrow \R$, c is a limit point of $\mathcal{A}$.\\
    (i) $\lim_{x\to\ c} f(x) = L\\ \Leftrightarrow$ 
    (ii) $\any \{x_n\} \subseteq \mathcal{A}$ s.t $x_n \neq c$ and $\lim_{n\to\infty} x_n = c \Rightarrow \lim_{n\to\infty} f(x_n) = L.$
\end{theorem}
\begin{proof}
    ($\Rightarrow$) \\
    Let $\{x_n\} \subseteq \mathcal{A}$ be given. $x_n \neq c$, $\epsilon > 0$ be given. \\
    Since 
    \[
    \lim_{x\to\ c} f(x) = c, \exist \delta s.t 0<|x-c|<\delta_0 \Rightarrow |f(x) -L| <\epsilon.
    \] \\
    Since 
    \[
    \lim_{x\to\infty} x_n = c, \exist N  s.t n>N \Rightarrow 0<|x_n -c| <\delta \text{ for the }\delta.
    \] \\
    Take $N = N_0$, assume $n>N$, then:\\
    \[
    0<|x_n-c|<\delta \Rightarrow |f(x_n) -L| <\epsilon.
    \]\\
    ($\Leftarrow$) \\
    Assume \[
    \exist \epsilon > 0 s.t \any \delta >0, \exist x_0 \in \mathcal{A}, 0<|x_0 -c| < \delta \text{ and } |f(x)-L| \geq \epsilon.
    \]
    Let $\delta = \cfrac{1}{n}$, 
    \[
    \exist x_n \neq c s.t 0<|x_n-c|<\cfrac{1}{n} \text{ and } |f(x_n) -L| \geq \epsilon, \any n \in \N,\] 
    thus, we make a sequence $\{x_n\}$,\\
    \uwave{thus, $x_n \rightarrow c$} $\Rightarrow |f(x_n) - L| <\epsilon$.
\end{proof}
\begin{theorem}[Algebraic Limit Theorem]
\label{Algebraic Limit Theorem}
    $f:\mathcal{A} \rightarrow \R$, c is a limit point on $\mathcal{A}$. \\
    $\exist \{ x_n\}, \{y_n\} \subseteq \mathcal{A}$, $x_n \neq c$, $y_n \neq c$, $\lim_{n\to\infty} x_n = \lim_{n\to\infty} y_n = c$, $\lim_{n\to\infty} f(x_n) \neq \lim_{x\to\infty} f(y_n) \Rightarrow \lim_{x\to\ c} f(x)$ D.N.E.
\end{theorem}
\begin{example}
    \[
        f(x) = 
    \begin{cases}
        \text{1 } x \in \Q \\
        \text{0 } x \in \Q^c
    \end{cases}\] \\
    Let $\delta = \cfrac{1}{n}$, then take $x_n \in \Q \cap V_\delta (a)$, $y_n \in \Q^c \cap V_\delta (a)$, then $x_n \rightarrow a$, $y_n \rightarrow a$, $\any a \in \R$, but $f(x_n) =1$, $f(y_n) = 0,$\\
    $\therefore \lim_{x\to\ a} f(x)$ D.N.E., $\any a \in \R$.
\end{example}
\begin{definition}
    $f:\mathcal{A}\rightarrow\R$ is continuous at a point of $\mathcal{A}$, $c \in \mathcal{A}$ if 
    \[
    \any \epsilon > 0, \exist \delta >0 s.t \uwave{|x-c|<\delta }\Rightarrow |f(x)-f(c)| < \epsilon.
    \]
    %\therefore 不需要 limit points
\end{definition}
\begin{definition}
    $f:\mathcal{A}\rightarrow\R$ is continuous at a point of $\mathcal{A}$ if f is continuous at every point of $\mathcal{A}$.
\end{definition}
\uwave{Criterion of Discontinuity} \\
Given $f: \mathcal{A}\rightarrow \R$, $c \in \mathcal{A}$ be a limit point of $\mathcal{A}$. \\
 \[
\exist \{x_n\} \subseteq \mathcal{A} s.t x_n \rightarrow c but f(x_n) \nrightarrow f(c).
\] \\
$\Rightarrow$ f is not c.t.s at c.
